%%%%%%%%%%%%%%%%
%% How To Use %%
%%%%%%%%%%%%%%%%
% 1. In main.tex -> Document style -> "Choose your template" section replace defalut closing page
%    \newcommand{\closingPage}{
	\newpage
	
	\section{Geräteliste}
	\textmd{Die Simulation wurden mit der aufgeführten Software durchgeführt:}
	\begin{table}[H]
		\centering\bgroup\def\arraystretch{1.5}%  1 is the default, change whatever you need
		\begin{tabular}{|p{6cm}|l|}
			\hline\rowcolor{lightgray}
			Bezeichnung     & Gerätenummer \\ \hline
			Matlab 	& 			   \\ \hline
		\end{tabular}
		\egroup
		\caption{Messgeräte}  %caption wird linkbündig geschrieben siehe package
		%\label{tab:Brückenspannung} %refernziert den Zählindex der Tabelle 
		% \ref{tab:Klemmspannung}
	\end{table}
	
	\section{Quellenverzeichnis}
	[1] Aufgabenblatt \Versuch:\\ \url{\LinkAufgabe}
	\listoffigures%Abbildungverzeichnis
	\listoftables%Tabellenverzeichnis
	
}         % Sub file provieds closing page template
% 2. In main.tex -> Document style -> "Definitions" section define 
%    \newcommand{\Versuch}{XYZ 1} 
%    \newcommand{\LinkAufgabe}{https://some/link/to/view.php?id=81238234}
% 3. In main.tex -> "Document content" use \closingPage to create the coling page  
%    \closingPage                                  % Create closing page


%%%%%%%%%%%%%%
%% Requires %%
%%%%%%%%%%%%%%
% Definitions:  
\newcommand{\versuch}{XYZ 1}                       % Attempt name            
\newcommand{\linkVersuch}{https:somelink.html}     % The link is inserted after the \versuch in the bibliography
% - \definecolor{lightgray}{rgb}{0.95,0.95,0.95}

% Packages:
% - \usepackage{float}                              % Placing graphics and tables with the [H] option 
% - \usepackage{adjustbox}                          % Fit tables to page width
% - \usepackage{colortbl}                           % Colorize tables


%%%%%%%%%%%%%%%%%%
%% Closing Page %%
%%%%%%%%%%%%%%%%%%
\newcommand{\closingPage}{                          % New command to create the coling page in the main.tex
    \newpage                                        % Start on a new page

    \section{Geräteliste}                           % Make a new section
    \textmd{Die Simulation wurden mit den aufgeführten Geräten durchgeführt:}
    \begin{table}[H]                                % create and place table environment 
        \centering                                  % align center 
        \begin{adjustbox}{max width=\textwidth}     % Adjusts the table to the maximum width 
            \def\arraystretch{1.5}                  % Distance between text and table  
            \begin{tabular}{|p{6cm}|l|}             % Create and place table
                \hline\rowcolor{lightgray}          % Change first row colour     
                Bezeichnung     & Gerätenummer \\ \hline
                                &              \\ \hline
                                &              \\ \hline
            \end{tabular}
        \end{adjustbox}
        \caption{Messgeräte}                         % Make caption 
        \label{tab:Messgeräte}                       % Label to reference table
    \end{table}
    
    \section{Quellenverzeichnis}                     % Make a new section
    [1] Aufgabenblatt \versuch: \url{\linkVersuch}   % Custom bibliography entry @TODO
    \listoffigures                                   % List of figures
    \listoftables                                    % List of tables
}